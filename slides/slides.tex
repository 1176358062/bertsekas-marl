
\documentclass{beamer}

\mode<presentation> {



%\usetheme{Darmstadt}
%\usetheme{Dresden}
\usetheme{Singapore}
%\usetheme{Szeged}
%\usetheme{Warsaw}

%\usecolortheme{albatross}
%\usecolortheme{beaver}
%\usecolortheme{beetle}
%\usecolortheme{crane}
%\usecolortheme{dolphin}
%\usecolortheme{dove}
%\usecolortheme{fly}
%\usecolortheme{lily}
%\usecolortheme{orchid}
%\usecolortheme{rose}
%\usecolortheme{seagull}
%\usecolortheme{seahorse}
%\usecolortheme{whale}
%\usecolortheme{wolverine}

%\setbeamertemplate{footline} % To remove the footer line in all slides uncomment this line
%\setbeamertemplate{footline}[page number] % To replace the footer line in all slides with a simple slide count uncomment this line

%\setbeamertemplate{navigation symbols}{} % To remove the navigation symbols from the bottom of all slides uncomment this line
}

\usepackage{graphicx} % Allows including images
\usepackage{booktabs} % Allows the use of \toprule, \midrule and \bottomrule in tables

\usepackage{hyperref}
\usepackage{graphicx}
\graphicspath{ {images/} }
\usepackage{xcolor}

%----------------------------------------------------------------------------------------
%	TITLE PAGE
%----------------------------------------------------------------------------------------

\title[MARL]{Multiagent Reinforcement Learning:
	Rollout and Policy Iteration
	(Bertsekas, 2020)} 
\author{Mikalai Korbit} % Your name

\institute[IMT] % Your institution as it will appear on the bottom of every slide, may be shorthand to save space
{
IMT School for Advanced Studies Lucca %\\ % Your institution for the title page
%\medskip
%\textit{todo@outlook.com} % Your email address
}
\date{\today} % Date, can be changed to a custom date

\begin{document}

\begin{frame}
\titlepage % Print the title page as the first slide
\end{frame}

\begin{frame}
\frametitle{Outline} % Table of contents slide, comment this block out to remove it
\tableofcontents % Throughout your presentation, if you choose to use \section{} and \subsection{} commands, these will automatically be printed on this slide as an overview of your presentation
\end{frame}



%----------------------------------------------------------------------------------------
%	PRESENTATION SLIDES
%----------------------------------------------------------------------------------------


%------------------------------------------------
\section{Introduction} 
%------------------------------------------------

\begin{frame}
	
\frametitle{Sidenote -- Path Integral Control}
TODO

\begin{itemize}
\item TODO

\end{itemize}

\end{frame}


%------------------------------------------------

\begin{frame}
\frametitle{Motivation}

\begin{block}{TODO}
TODO. 
\end{block}


\end{frame}


%------------------------------------------------

\begin{frame}
	\frametitle{Outline}
	\begin{block}{TODO}
		TODO
	\end{block}

\end{frame}


%------------------------------------------------






%------------------------------------------------
\section{Brief Overveiw of MARL}
%------------------------------------------------



\begin{frame}
\frametitle{Reinforcement Learning Problem}
\begin{itemize}
	\item Learning how to map situations to actions 
	\item Trial-and-error search
	\item Delayed feedback
	\item Trade-off between exploration and exploitation
	\item Sequential decision making
	\item Agent's actions affect the subsequent data it receives
\end{itemize}
\end{frame}



%------------------------------------------------
\section{Multiagent Rollout}
%------------------------------------------------

\begin{frame}
	\frametitle{Reinforcement Learning Problem}
	\begin{itemize}
		\item Learning how to map situations to actions 
		\item Trial-and-error search
		\item Delayed feedback
		\item Trade-off between exploration and exploitation
		\item Sequential decision making
		\item Agent's actions affect the subsequent data it receives
	\end{itemize}
\end{frame}


%------------------------------------------------
\section{Extensions}
%------------------------------------------------

\begin{frame}
	\frametitle{Reinforcement Learning Problem}
	\begin{itemize}
		\item Learning how to map situations to actions 
		\item Trial-and-error search
		\item Delayed feedback
		\item Trade-off between exploration and exploitation
		\item Sequential decision making
		\item Agent's actions affect the subsequent data it receives
	\end{itemize}
\end{frame}





%------------------------------------------------

\section{Conclusion}

\begin{frame}
\frametitle{Conclusion}
\begin{itemize}
	\item RL methods can be applicable to
	a wide variety of problems
	
	\item Out-of-the-box models work but 
	require fine-tuning and take 
	longer to converge
	
	\item Simple methods like state discretization
	are worth exploring when training speed and
	solution complexity are of the essence 
	
\end{itemize}
\end{frame}


\begin{frame}
\frametitle{References}
\footnotesize{
\begin{thebibliography}{5} 
	
\bibitem{rl1}\label{rl_book}
Reinforcement learning: an introduction, 2nd Edition. 
Richard S. Sutton, Andrew G. Barto. 

\bibitem{rl2}\label{ucl_course}
Reinforcement learning lectures by David Silver. UCL.
\url{http://www0.cs.ucl.ac.uk/staff/d.silver/web/Teaching.html} 

\bibitem{rl3}\label{deepmind}
Playing Atari with Deep Reinforcement Learning. Mnih et al.
\url{https://arxiv.org/abs/1312.5602}


\end{thebibliography}
}
\end{frame}

%------------------------------------------------

\begin{frame}

\begin{center}
	\Huge Thanks for
	\\
	your attention!
\end{center}

\end{frame}

%----------------------------------------------------------------------------------------

\end{document}